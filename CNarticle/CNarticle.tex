\documentclass{ctexart}
\setlength{\headheight}{15pt}
\usepackage{fancyhdr}
\usepackage{amsmath,amsthm,amsfonts,amssymb,geometry} 
\usepackage{caption,graphicx,subfig}
\usepackage{float}
\usepackage{tcolorbox}
\usepackage[cache=false]{minted}
\usepackage{algorithm,algorithmic}
\usepackage[colorlinks=true]{hyperref}
\usepackage[title]{appendix}
\geometry{a4paper,left=3.18cm,right=3.18cm,top=2.54cm,bottom=2.54cm}

%设置页眉
\pagestyle{fancy}
\fancyhf{} 
\fancyhead[C]{\leftmark}
\fancyfoot[C]{\thepage}
\renewcommand{\headrulewidth}{0.3mm} 
\renewcommand{\footrulewidth}{0mm}

%设置定理环境
\newtheorem{theorem}{定理}[section]

%标题作者日期
\title{\textbf{中文文章模板}}
\author{李想}
\date{}
\begin{document}
\maketitle

%目录
\pagenumbering{roman}
\tableofcontents
\newpage

%正文
\pagenumbering{arabic}
\setcounter{page}{1}

\section{线性代数}

%定理边框
\begin{tcolorbox}
[colframe=cyan!40!black,
title={\begin{theorem}
[秩-零化度]\end{theorem}}]
线性空间$V$的维数
\footnote{维数是指...}
等于...
\end{tcolorbox}

\begin{proof}
    秩\footnote{秩是...}
\end{proof}
根据 ...\cite{ref1}

\newpage

%代码
\section{代码}
\begin{algorithm}[!h]
	\caption{PARTITION$(A,p,r)$}%算法标题
	\textbf{Input: }{$\mathbf{x},
    \mathbf{y}\in\mathbb{R}^n$}\\
    \textbf{Output:}
    {$\mathbf{z}=\mathbf{x}
    \mathbf{y}^T\in\mathbb{R}^{n\times n}$}
    \begin{algorithmic}[1]%一行一个标行号
		\STATE $i=p$
		\FOR{$j=p$ to $r$}
		\IF{$A[j]<=0$}
		\STATE $swap(A[i],A[j])$
		\STATE $i=i+1$
		\ENDIF
		\ENDFOR
    \end{algorithmic}
\end{algorithm}

\begin{minted}[linenos,frame=lines]{c++}
    int main() {
    printf("hello, world");
    return 0;
    }
\end{minted}

\newpage

%附录
\begin{appendices}
\section{多重线性代数}
\end{appendices}
\newpage

%参考文献
\begin{thebibliography}{99}
\bibitem{ref1}张,
\end{thebibliography}

\end{document}